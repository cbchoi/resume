%%%%%%%%%%%%%%%%%%%%%%%%%%%%%%%%%%%%%%%%%
% Cies Resume/CV
% LaTeX Template
% Version 1.0 (10/2/14)
%
% This template has been downloaded from:
% http://www.LaTeXTemplates.com
%
% Original author:
% Cies Breijs (cies@kde.nl)
% https://\textscgithub.com/cies/resume with extensive modifications by:
% Vel (vel@latextemplates.com)
%
% License:
% CC BY-NC-SA 3.0 (http://creativecommons.org/licenses/by-nc-sa/3.0/)
%
% Modifed by:
% Changbeom Choi (cbchoi82@gmail.com)
% Version 1.1 (2014.02.25)
%%%%%%%%%%%%%%%%%%%%%%%%%%%%%%%%%%%%%%%%%

%----------------------------------------------------------------------------------------
%	DOCUMENT CONFIGURATIONS
%----------------------------------------------------------------------------------------
\documentclass[english,representative]{resume_structure}
%\documentclass[english,representative]{resume_structure}

%----------------------------------------------------------------------------------------
% BIO INFORMATION
%----------------------------------------------------------------------------------------
\ResumeName{최창범}{崔暢汎}{Changbeom Choi}
\ResumeBirth{1982.10.30}
\ResumeEmail{cbchoi@handong.edu}
\ResumeMobile{82-10-2735-5723}
\ResumeHomepage{http://pro.handong.edu/cbchoi/}
\ResumeAddress{유성구 엑스포로 448 (전민동, 엑스포 아파트) 405동 403호}{(305-761) Expo Apt. 405-403, 448 Expo-ro, Yu-sung gu }
\ResumeCity{대전광역시}{Daejeon}
\ResumeNation{대한민국}{South Korea}

\begin{document} 
%----------------------------------------------------------------------------------------
%	NAME AND CONTACT INFORMATION
%----------------------------------------------------------------------------------------

\makeheader

%----------------------------------------------------------------------------------------
%	EXPERIENCE SECTION
%----------------------------------------------------------------------------------------
\begin{Education} % Top level section

\ResumeSectionWithSubSection % Employer name which can include a hyperlink and location/URL on the right side of the page
{경희대학교}{Kyunghee University}
{수원시}{Suwon, South Korea} 
{
  \ResumeSubSection % Job title entry for the current employer
  {전공: 컴퓨터 공학}{Major: Computer Engineering}
  {2001 -- 2005}
  {
  \begin{itemize}
    \TypeItem{rep}{학사 학위 취득, 2005.02; 학점 4.19/4.3}
    \TypeItem{full}{주 관심사: 컴퓨터 프로그래밍}
    \TypeItem{full}{관련 과목: 객체 지향 프로그래밍;윈도우 프로그래밍;인터넷 프로그래밍}
  \end{itemize}
  }{
    \begin{itemize}
    \TypeItem{rep}{ B.S. degree, February 2005; GPA 4.19/4.3}
    \TypeItem{full}{ Concentration: Computer Programming}
    \TypeItem{full}{ Relevant Courses: Object-Oriented Programming; Windows Programming; Internet Programming}
  \end{itemize}
  }
}
%------------------------------------------------

\ResumeSectionWithSubSection % Employer name which can include a hyperlink and location/URL on the right side of the page
{한국과학기술원}{Korea Advanced Institute of Science and Technology}
{대전시, 대한민국}{Daejeon, South Korea} 
{
  \ResumeSubSection % Job title entry for the current employer
  {전공: 컴퓨터 과학}{Major: Computer Science}
  {2005 -- 2007}
  {
  \begin{itemize}
    \TypeItem{rep}{석사 학위 취득, 2007.02; 학점 3.75/4.3}
    \TypeItem{full}{주 관심사: 알고리즘; 그래프 이론; 경영전략}
    \TypeItem{full}{관련 과목: 알고리즘 설계 및 분석; 그래프 이론; 기업가 정신 및 경영전략}
  \end{itemize}
  }{
  \begin{itemize}
    \TypeItem{rep} {M.S. degree, February 2007; GPA 3.75/4.3}
    \TypeItem{full} {Concentration: Algorithms; Graph Theory}
    \TypeItem{full} {Relevant Courses: Algorithm Design and Analysis; Graph Theory; Entrepreneurship and Business Strategies}
  \end{itemize}
  }
}
%------------------------------------------------
\ResumeSectionWithSubSection % Employer name which can include a hyperlink and location/URL on the right side of the page
{한국과학기술원}{Korea Advanced Institute of Science and Technology}
{대전시, 대한민국}{Daejeon, South Korea} 
{
  \ResumeSubSection % Job title entry for the current employer
  {전공: 전기 및 전자공학과}{Major: Electrical Engineering}
  {2007 -- 2014}
  {
  \begin{itemize}
    \TypeItem{rep}{박사 학위 취득 예정, 2014.08; 학점 3.65/4.3}
    \TypeItem{full}{주 관심사: 모델링 및 시뮬레이션}
    \TypeItem{full}{관련 과목: 이산 사건 시스템 모델링 및 시뮬레이션}
  \end{itemize}
  }{
  \begin{itemize}
    \TypeItem{rep}{Candidate for Doctor of Philosophy degree, August 2014; GPA 3.65/4.3}
    \TypeItem{full}{Concentration: Modeling \& Simulation; Software Engineering}
    \TypeItem{full}{Relevant Courses: Discrete Event System Modeling \& Simulation}
  \end{itemize}
  }
}
\end{Education}

%----------------------------------------------------------------------------------------
%    WORK EXPERIENCE
%----------------------------------------------------------------------------------------
\begin{Work}
\ResumeSectionWithSubSection % Employer name which can include a hyperlink and location/URL on the right side of the page
{\href{http://www.handong.edu}{한동대학교}}{\href{http://www.handong.edu}{School of Global Entrepreneurship and ICT}}
{\textsc{포항시, 대한민국}}{\textsc{Pohang, South Korea}} {
  \ResumeSubSection % Job title entry for the current employer
    {조교수}{Assistant Professor}
    {2016.09 -- Current}
    {
    \begin{itemize}
      \item PRIME 사업단 교수학생지원실 실장 (2016.08 -- Current)
    \end{itemize}
    }
    {
    \begin{itemize}
      \item Director of Faculty and Student Support, PRIME Project (2016.08 -- Current)
    \end{itemize}
    }
}
\ResumeSectionWithSubSection % Employer name which can include a hyperlink and location/URL on the right side of the page
{\href{http://www.handong.edu}{한동대학교}}{\href{http://www.handong.edu}{School of Creative Convergence Study}}
{\textsc{포항시, 대한민국}}{\textsc{Pohang, South Korea}} {
  \ResumeSubSection % Job title entry for the current employer
    {조교수}{Assistant Professor}
    {2014.09 -- 2016.08}
    {
    \begin{itemize}
      \item 제 66차 UN NGO DPI 컨퍼런스 청년위원회 위원 (2016.05.30 -- 2016.06.01)
    \end{itemize}
    }
    {
    \begin{itemize}
      \item 66th UN NGO DPI Conference Youth Comittee member (2016.05.30 -- 2016.06.01)
    \end{itemize}
    }
}
\ResumeSectionWithSubSection % Employer name which can include a hyperlink and location/URL on the right side of the page
{\href{http://cs.kaist.ac.kr}{전산학과}}{\href{http://cs.kaist.ac.kr}{Dept. of Computer Science}}
{\textsc{대전, KAIST, 대한민국}}{\textsc{Daejoen, KAIST, South Korea}} {
  \ResumeSubSection % Job title entry for the current employer
    {별정직 연구원}{Researcher}
    {2007.03 -- 2007.08}
    {
    \begin{itemize}
      \item 모바일 어플리케이션의 보안 취약성을 진단하기 위하여 취약성 분류 (스마트폰이 대중화되기 이전)
      \item 모바일 어플리케이션의 취약성을 실시간으로 모니터링하고 진단하기 위하여 실시간 검증 기법의 도입
    \end{itemize}
    }
    {
    \begin{itemize}
      \item Categorized the security vulnerablities for mobile applications before the smart phones are popularized 
      \item Introduced the Run-time Verification technology to monitor and assess the vulnerability of mobile applications in run-time
    \end{itemize}
    }
}
\end{Work}

%----------------------------------------------------------------------------------------
%	PROJECT SECTION
%----------------------------------------------------------------------------------------

\begin{Project}
  
\TypeSectionWithSubSection {full}
{실시간 검증 기반의 모바일 코드 취약성 분석 및 평가}{Mobile Code Vulnerability Analysis and Assessment based on Real-time Verification}
{}{} 
{
\ResumeSubSection % Job title entry for the current employer
    {직급: 참여연구원 (역할:제 4 세부과제 실무책임자)}{Researcher (Position: Hands on Worker)}
    {2007.02 -- 2007.12}
    {
      \begin{itemize}
        \item 본 연구는 (재)사이버기술연구소의 지원 및 협업으로 수행되었음
        \item 보안 전문가와의 협업을 통하여 응용 프로그램으로 인한 PDA 및 스마트폰의 보안 취약성 분류
        \item 응용 프로그램의 API 호출을 실시간으로 모니터링하여 실시간 검증 기반의 보안 취약성 보고 시스템 개발
      \end{itemize}
    }
    {
      \begin{itemize}
        \item This project was supported and cooperated by Cyber Technology Research Foundation
        \item Categorized security vulnerabilities of PDA and smart-phone caused by mobile application (cooperated with security specialist)
        \item Developed vulnerability report system based on the Run-time Verification 
      \end{itemize}
    }
}

\TypeSectionWithSubSection {rep}
{다대다 교전 시뮬레이션 구축 기법 연구}{Research of Many to Many Combat Simulation System Construction}
{}{} 
{
\ResumeSubSection % Job title entry for the current employer
    {직급: 참여연구원 (역할:실무 책임자)}{Position: Researcher (Role: Hands on Worker)}
    {2007.09 -- 2009.12}
    {
      \begin{itemize}
        \item 국방과학연구소의 연구원과 삼성탈레스의 개발진과 협업으로 차기호위순양함의 지휘통제장치 검증 프레임워크 개발
        \item 단독 시뮬레이션 및 연동 시뮬레이션 기능을 구현하여 시뮬레이션 기반의 획득 수행
      \end{itemize}
    }
    {
      \begin{itemize}
        \item Developed verification framework of command and control system for next generation cruiser with researchers of the Agency for Defense Development and development team of the Samsung Thales
        \item Developed standalone simulation and interoperation simulation features to assist the researchers of the agency
      \end{itemize}
    }
}

\TypeSectionWithSubSection {rep}
{사용자 중심의 개방형 및 진화형 현실 모사 가상 세계 프레임워크 기술 개발}{Development of Virtual World Framework Technology: User Created, Evolutionary, Realistic Simulation}
{}{} 
{
\ResumeSubSection % Job title entry for the current employer
    {직급: 참여연구원 (역할:콘소시엄 내 기술총괄)}{Researcher (Position: Chief Technical Officer at the consortium)}
    {2009.06 -- 2013.02}
    {
      \begin{itemize}
        \item 삼성전자, SBS 콘텐츠허브, 전자부품연구원 및 타 중소기업과의 협업
        \item 이산사건시스템 형식론 기반의 지능형 가상에이전트 프레임워크 개발
        \item 콘소시업의 요소기술의 통합 관리
      \end{itemize}
    }
    {
      \begin{itemize}
        \item Cooperated with the researcher of Samsung, SBS Contents Hub, and other venture company
        \item Developed intelligent virtual agent framework to assist the developer to build a services easily
        \item Analyzed the technologies of the consortium members and establish an integration plan
      \end{itemize}
    }
}

\TypeSectionWithSubSection {full}
{관점지향 프로그래밍 기술을 사용한 소프트웨어 정형 검증 기법}{Software Formal Verification Method using Aspect Oriented Programing Technique}
{}{} 
{
\ResumeSubSection % Job title entry for the current employer
    {연구책임자}{Chief of Research}
    {2009.01 -- 2009.12}
    {
      \begin{itemize}
        \item KAIST High Risk High Return 프로그램에 선정되어 학생으로서 연구책임자가 됨
        \item 관점 지향 프로그래밍 기법을 적용하여 소프웨어가 요구사항을 만족하는 지를 검증 방법론 제시
      \end{itemize}
    }
    {
      \begin{itemize}
        \item Selected by High Risk High Return Program of KAIST; Became a Chief of Research as a Ph.D student
        \item Proposed a verfication methodology based on the aspect oriented programming to solve the verification problem of software
      \end{itemize}
    }
}

\TypeSectionWithSubSection {full}
{잠수함 전투체계 효과도 분석을 위한 시뮬레이션 구축 기법 연구}{Research of Simulation Construction Technique for Effective Analysis of Submarine Battle System}
{}{} 
{
\ResumeSubSection % Job title entry for the current employer
    {직급: 참여연구원(역할:시뮬레이터 개발팀장)}{Researcher (Position: Chief of Technical Officer)}
    {2010.01 -- 2012.12}
    {
      \begin{itemize}
       \item 이산사건시스템 형식론 기반의 해양 무기체계의 모델 템플릿 개발
       \item 해양 무기체계 모델 템플릿을 바탕으로 한 워게임 시뮬레이터 개발
      \end{itemize}
    }
    {
      \begin{itemize}
        \item Developed a model templates of the naval warfare system based on the Discrete Event System Formalism
        \item Developed wargame simulator based on the template of the naval warfare model
     \end{itemize}
    }
}

\TypeSectionWithSubSection {full}
{공학급 교전급 모델간 연동 연구}{Research of Interoperation between Engineering Model and Engagement Model}
{}{} 
{
\ResumeSubSection % Job title entry for the current employer
    {역할:프로젝트 기술 자문}{Position: Researcher (Role:Technical Advisor)}
    {2011.02 -- 2013.12}
    {
      \begin{itemize}
        \item 공학급 교전급 연동 시뮬레이션 체계 개발
        \item 효율적인 실험 결과 획득을 위한 연동형 실험환경 개발 및 구축
      \end{itemize}
   }
    {
      \begin{itemize}
       \item Developed experimental environment for interoperation systems
       \item Developed interoperation system of the engineering model and engagement models
      \end{itemize}
    }
}

\TypeSectionWithSubSection {rep}
{KCTC 피해 평가 모델 개발}{Development of Damage Assessment Model of Korea Combat Training Center(KCTC)}
{}{} 
{
\ResumeSubSection % Job title entry for the current employer
    {직급: 참여연구원 (역할:실무 책임자)}{Position: Researcher (Role: Hands on Worker)}
    {2012.04 -- 2012.09}
    {
      \begin{itemize}
        \item 기존 육군과학화훈련장의 피해평가 모델 분석
        \item 다양한 피해평가 모델의 제안
        \item 육군과학화훈련장, 방위사업청, 개발 콘소시움간 의견 조율
      \end{itemize}
    }
    {
      \begin{itemize}
        \item Analyzed the existing damage assessment model of KCTC
        \item Proposed various damage assessment models based on the research
        \item Opinion coordination among military officers, researchers of Defense Acquisition Program Administration, and developement consortium
      \end{itemize}
    }
}

\TypeSectionWithSubSection {full}
{하둡 시스템의 모델링 및 시뮬레이션 프레임워크 개발}{Development of Modeling and Simulation Framework of Hadoop System}
{}{} 
{
\ResumeSubSection % Job title entry for the current employer
    {직급: 참여연구원 (역할:프로젝트 기술 자문)}{Position: Researcher (Role:Technical Advisor)}
    {2012.04 -- 2012.09}
    {
      \begin{itemize}
        \item 하둡 시스템의 모델링 및 시뮬레이션을 위한 사용자 인터페이스 체계 제안
        \item 하둡 시스템의 모델링 및 시뮬레이션을 위한 사용자 인터페이스 체계 구현
      \end{itemize}
    }
    {
      \begin{itemize}
        \item Proposed User Interface(UI) system of the Hadoop Simulation Framework to assist the stakeholder
        \item Developed the UI system using C\#
      \end{itemize}
    }
}

\TypeSectionWithSubSection {rep}
{ExtendSim과 Matlab 및 SIMULINK 연동 체계에 대한 연구}{Research on the Interoperation System between ExtendSim and Matlab/Simulink}
{}{} 
{
\ResumeSubSection % Job title entry for the current employer
    {직급: 연구책임자 (역할:프로젝트 총괄)}{Position: Chief of Research (Role: Generalize Development)}
    {2014.09 -- 2015.02}
    {
      \begin{itemize}
        \item ExtendSim과 Matlab/Simulink 사이의 연동 데이터 표준 정립
        \item C++ 기반의 연동 프로토타입 개발
      \end{itemize}
    }
    {
      \begin{itemize}
        \item Proposed Standard Data Interoperation Formats for ExtendSim and Matlab/Simulink
        \item Developed the prototype of the interoperation system using C++
      \end{itemize}
    }
}

\TypeSectionWithSubSection {rep}
{파이로 시뮬레이션 ExtendSim과 Matlab 모델의 연동 기술 개발}{Development of simulation environment for interoperation between Pyro simulation model using ExtendSim and Matlab simulation model}
{}{} 
{
\ResumeSubSection % Job title entry for the current employer
    {직급: 연구책임자 (역할:프로젝트 총괄)}{Position: Chief of Research (Role: Generalize Development)}
    {2015.03 -- 2016.02}
    {
      \begin{itemize}
        \item ExtendSim과 Matlab/Simulink 사이의 연동 데이터 표준 정립 및 연동 알고리즘 개발
        \item 다양한 수준의 시뮬레이션 간 연동을 위한 연동 모듈 개발 
      \end{itemize}
    }
    {
      \begin{itemize}
        \item Proposed Standard Data Interoperation System for ExtendSim and Matlab/Simulink
        \item Developed the interoperation system using C++
      \end{itemize}
    }
}

\TypeSectionWithSubSection {rep}
{파이로 시설의 안전 조치 모듈 연계 M\&S 체계 설계 및 구축 계획}{Modleing and simulation environment for safeguard module interoperation of Pyro facility}
{}{} 
{
\ResumeSubSection % Job title entry for the current employer
    {직급: 연구책임자 (역할:프로젝트 총괄)}{Position: Chief of Research (Role: Generalize Development)}
    {2016.03 -- 2017.02}
    {
      \begin{itemize}
        \item 파이로 시설의 안전조치 모듈과 기존 M\&S 체계를 연동하기 위한 연동 아키텍처 개발
        \item 파이로 시뮬레이션과 안전조치모듈 간의 연동 시스템 개발
      \end{itemize}
    }
    {
      \begin{itemize}
        \item Proposing interoperation architecture between pyro simulation model and safeguard simulation model
        \item Developing interoperation system of the pyro simulation and safeguard simulation
      \end{itemize}
    }
}
\end{Project}

%----------------------------------------------------------------------------------------
%	PUBLICATION SECTION
%----------------------------------------------------------------------------------------
\begin{Publication} % Top level section

\TypeSectionWithSubSection {rep}
{국\ 제\ 저\ 널}{International Journal}
{}{} 
{
\begin{itemize}
  \TypeItem{full}{Jeong Hoon Kim, \underline{Chang Beom Choi}, and Tag Gon Kim, "Battle Experiments of Naval Air Defense with Discrete Event System-based Mission-level Modeling and Simulations," The Journal of Defense Modeling and Simulation: Applications, Methodology, Technology, Vol. 8, No. 3, pp. 173 - 187, July., 2011.}
  \TypeItem{rep}{Tag Gon Kim, Chang Ho Sung, Su-Youn Hong, Jeong Hee Hong, \underline{Chang Beom Choi}, Jeong Hoon Kim, Kyung Min Seo, and Jang Won Bae, "DEVSim++ Toolset for Defense Modeling and Simulation and Interoperation," The Journal of Defense Modeling and Simulation: Applications, Methodology, Technology, Vol. 8, No. 3, pp. 129 - 142, July., 2011.}
  \TypeItem{rep}{\underline{Chang Beom Choi}, Kyung-Min Seo, Tag Gon Kim, "DEXSim: an experimental environment for distributed execution of replicated simulators using a concept of single simulation multiple scenarios," SIMULATION: Transaction of The Society for Modeling and Simulation International, Feb., 2014. \emph{doi: 10.1177/0037549713520251}}
  \TypeItem{full} {Kyung-Min Seo, \underline{Chang Beom Choi}, Tag Gon Kim, and Jung Hoon Kim, "DEVS-based combat modeling for engagement-level defense simulation," SIMULATION: Transaction of The Society for Modeling and Simulation International, Accepted to be published, Feb., 2014.}
  \TypeItem{full} {Minwook Yu, \underline{Chang Beom Choi}, and Tag Gon Kim, "High-Level Architecture Service Management for the Interoperation of Federations," SIMULATION: Transaction of The Society for Modeling and Simulation International, vol.91, pp. 566~590, 2015.}
  \TypeItem{full} {I.-J. Kim, \underline{C. Choi}, and S.-H. Lee, "Improving discrimination ability of convolutional neural networks by hybrid learning," International Journal on Document Analysis and Recognition (IJDAR), vol. 19, no. 1, p. 1, 2015.}
\end{itemize}
}

\TypeSectionWithSubSection {rep}
{국\ 제\ 학\ 회}{International Conference}
{}{} 
{
\begin{itemize}
  \TypeItem{full}{Jong Hyuk Byun, \underline{Chang Beom Choi} and Tag Gon Kim, "Verification of the DEVS Model Implementation using Aspect Embedded DEVS," in Proceedings of 2009 Spring Simulation MultiConf., San Diego, CA, USA, Mar., 2009}
  \TypeItem{full} {JeongHoon. Kim, \underline{Chang Beom Choi}, IlChul Moon, and TagGon Kim, "DEVS based Validation of Warship Anti-Air Defense Doctrin," in Proceedings of 2010 Spring Simulation MultiConf.}
  \TypeItem{rep} {\underline{Chang Beom Choi}, JangWon Bae and Tag Gon Kim, "Challenges of Teaching Modeling and Simulation Theory to the Domain Experts in a Blended Learning Environment," in Proceedings of International Conference on Information Technology Based Higher Education and Training, Urgup, Turkey, Apr., 2010.}
  \TypeItem{full} {\underline{Chang Beom Choi}, Jangwon Bae, Minwook Yoo, Tag Gon Kim, and Soohan Kim, "Verification of Combat System using Discrete Event System Specification Formalism," in Proceedings of DHSS 2011, Italy, Sept., 2011}
  \TypeItem{full} {Kyung-Min Seo, \underline{Chang Beom Choi}, Jung Hoon Kim, and Tag Gon Kim, “Interface Forms for an Underwater Warfare Simulation Environment,” in Proceedings of DHSS 2011, Italy, Sept., 2011}
  \TypeItem{full} {\underline{Chang Beom Choi}, Se Jung Kwon, Tag Gon Kim, Jae Hyun Lim, Dong-Hyun Baek, and Soohan Kim, "Extendable Simulation Framework for Virtual World Environment based on the DEVS Formalism," in Proceedings of AsiaSim 2011, Korea, Nov., 2011.}
  \TypeItem{full} {Byeong Soo Kim, \underline{Chang Beom Choi}, and Tag Gon Kim, "Multifaceted Modeling and Simulation Framework for System of Systems Using HLA/RTI," in Proceedings of 2013 Spring Simulation Multiconference, 16th Communications and Networking Symposium (CNS), San Diego, CA, USA, April, 2013.}
  \TypeItem{rep} {\underline{Chang Beom Choi}, Moon-Gi Seok, Seon Han Choi, Tag Gon Kim, and Soohan Kim "Serious game development methodology by via interoperation between a constructive simulator and a game application using HLA/RTI," in Proceedings of the 10th International Multidisciplinary Modeling \& Simulation Multiconference 2013, Athena, Greece, Sep., 2013.}
  \TypeItem{full} {Min Wook Yoo, \underline{Chang Beom Choi} and Tag Gon Kim "Time Management In Hierarchical Federation Using RTI-RTI Interoperation," in Proceedings of 2013 Winter Simulation Multiconference, Washington D.C., USA, Dec., 2013.}
  \TypeItem{full} {BitNal Kim, OnYu Kang, Seonwha Baek, Yonghyun Shim,\underline{Changbeom Choi} "Agent-Based Simulation for Taxi and Customer Matching," in Proceedings of 2015 Asia Simualtion Conference, Jeju, Korea, Nov., 2015.}
  \TypeItem{full} {Yun Jong Kim, Ji Yong Yang,Joong sup Lee, and \underline{Changbeom Choi} "DEVS based simulation environment for Mobile Application," in Proceedings of 2015 Asia Simualtion Conference, Jeju, Korea, Nov., 2015.}
\end{itemize}
}
\end{Publication}

%----------------------------------------------------------------------------------------
%	Honors and Awards SECTION
%----------------------------------------------------------------------------------------
\begin{Others} % Top level section

\ResumeSubSection % Job title entry for the current employer
    {기\ 타}{Honors}
    {}
    {
      \begin{itemize}
         \item 경희대학교 수석 졸업
      \end{itemize}
    }
    {
      \begin{itemize}
        \item Graduate summa cum laude from Kyunghee University
      \end{itemize}
    }

\ResumeSubSection % Job title entry for the current employer
    {수\ 상}{Awards}
    {}
    {
      \begin{itemize}
          \item 경희대학교 총장상 수상 (2002.09, 2005.02) 
          \item The First International Bernard P. Zeigler DEVS Modeling and Simulation Award, Track II: Commercial DEVS application, First Prize. 2011
          \item 최우수 논문상 the International Defense and Homeland Security Simulation Workshop (the 10th International Multidisciplinary Modeling \& Simulation Multiconference 2013) 
      \end{itemize}
    }
    {
      \begin{itemize}
        \item The university president awards in Kyunghee University (2002.09, 2005.02) 
        \item The First International Bernard P. Zeigler DEVS Modeling and Simulation Award, Track II: Commercial DEVS application, First Prize. 2011
        \item Best Paper Awards of the International Defense and Homeland Security Simulation Workshop (the 10th International Multidisciplinary Modeling \& Simulation Multiconference 2013) 
      \end{itemize}
    }

\ResumeSubSection % Job title entry for the current employer
    {특\ 허}{Patent}
    {}
    {
      \begin{itemize}
        \item 이산사건 형식론 기반 시뮬레이션 모델의 계층형 관리방법, Patent Number: 10-1362383, 2014.02.06
        \item 모바일 기기를 위한 분산 DEVS 시뮬레이션 아키텍쳐, 10-1358075, 2014.01
        \item 효율적 시뮬레이션 정보 수집을 위한 실험 틀 및 이를 이용한 시뮬레이션 방법 및 시스템, 10-1158637, 2012.06
        \item 플러그인 기반 계층적 지능형 가상 에이전트 프레임워크, 10-1153316, 2014.05
      \end{itemize}
    }
    {
      \begin{itemize}
        \item Hierarchical Management Method of DEVS Based Simulation Model, Patent Number: 10-1362383, 2014.02.06
        \item Distributed DEVS Simulation Architecture for Mobile Devices, 10-1358075, 2014.01
        \item Experimental Frame for efficient simulation data collection, and simulation method and system using it, 10-1158637, 2012.06
        \item Hierarchical Intelligent Virtual Agent Framework based on plug-in, 10-1153316, 2014.05
      \end{itemize}
    }
%------------------------------------------------
\end{Others}

%----------------------------------------------------------------------------------------
%	SKILLS SECTION
%----------------------------------------------------------------------------------------

%\roottitle{Skills and Other Information} % Top level section
%\headedsubpublication % Job title entry for the current employer
%{Skills}
%{
%\begin{itemize}
%  \item Microsoft Office literacy
%  \item C/C++, C\#, Java, XML, HTML, AVR Programming, \LaTeX \ literacy
%\end{itemize}
%}
%\headedsubpublication % Job title entry for the current employer
%{Languages}
%{
%\begin{itemize}
%  \item Working knowledge of English
%  \item TOEIC 825 (Listening: 480, Reading: 325)
%\end{itemize}
%}
%\headedsubpublication % Job title entry for the current employer
%{Other Information}
%{
%\begin{itemize}
%  \item Completed Military Service as an exception and finished military duties at KAIST 
%\end{itemize}
%}
%\spacedhrule{1.6em}{-0.4em} % Horizontal rule - the first bracket is whitespace before and the second is after

%----------------------------------------------------------------------------------------
%	Reference SECTION
%----------------------------------------------------------------------------------------

%\begin{Reference} % Top level section

%\ResumeSectionWithSubSection % Special section that has an inline header with a 'hanging' paragraph
%{김탁곤}{Tag Gon Kim}
%{카이스트 전기 및 전자 공학과 교수}{Professor at Dept. of Electrical Engineering, KAIST}
%{ 
%  \ResumeSubSection % Special section that has an inline header with a 'hanging' paragraph
%  {}{}
%  {}
%  {전화번호:\textsmaller{+}82-42-350-5454, 이메일: \href{tkim@ee.kaist.ac.kr}{tkim@ee.kaist.ac.kr}}
%  {Tel:\textsmaller{+}82-42-350-5454, E-mail: \href{tkim@ee.kaist.ac.kr}{tkim@ee.kaist.ac.kr}}
%}
%
%\ResumeSectionWithSubSection % Special section that has an inline header with a 'hanging' paragraph
%{김수한}{Soohan Kim}
%{삼성전자 영상 디스플레이 사업부 수석 연구원}{Principal Engineer of Samsung Electronics Co. HQ.}
%{ 
%  \ResumeSubSection % Special section that has an inline header with a 'hanging' paragraph
%  {}{}
%  {}
%  {전화번호: \textsmaller{+}82-10-9530-4112, 이메일: \href{mailto:ksoohan@samsung.com}{ksoohan@samsung.com}}
%  {Tel: \textsmaller{+}82-10-9530-4112, E-mail: \href{mailto:ksoohan@samsung.com}{ksoohan@samsung.com}}
%}
%%----------------------------------------------------------------------------------------
%\end{Reference}
\end{document}